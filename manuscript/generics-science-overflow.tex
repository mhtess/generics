

These strong inferences are puzzling when examined next to the prevalence required for a generic to be true. 
As we saw in Expt.~1, generics can be true for a range of prevalence levels. 
This phenomenon clearly distinguishes generic statements from quantified statements.
``All lorches have purple feathers.'' is true only when 100\% of lorches have purple feathers. 
Similarly, upon hearing such an utterance, one is likely to infer that 100\% of lorches have purple feathers.  
This symmetric relationship holds with the quantifier ``most'', but generic utterances don't behave in this way \cite{Cimpian2010}.
Generic statements are judged true for a wide range of prevalence levels, but upon hearing a generic utterance, participants were wont to infer that \emph{almost all} of the category have the property. 
The implications of generic statements go far beyond the evidence needed to accept them as valid utterances.

The strength of the inference also depends on the property in question.
\citeA{Cimpian2010} observed that predicating bare plurals with accidental properties (e.g. ``Lorches have muddy feathers.'') significantly reduced participants' judgments about the implied prevalence of these statements.

Expt.~2 attempted to replicate, extend and explain the findings of \citeA{Cimpian2010}: that there is an asymmetry between truth conditions and implied prevalence of the generic and that this asymmetry is sensitive to the type of property predicated. 
In Expt.~2a, we measure participants' beliefs about the distribution of these types of properties using a paradigm generalized from Expt.~1a. 
In Expt.~2b, we replicate and extend \citeauthor{Cimpian2010}'s findings to different classes of properties for which the prevalence priors differed. 
We show that the language understanding model predicts this asymmetry between truth conditions and implications, and that this asymmetry can be manipulated by differences in the prevalence distributions between properties.








The largest deviations occur for the items ``Robins carry malaria.'' and ``Sharks have manes.''. 
Neither of these is true, and participants judge them to be as such.
The model predictions these are bad but better than items such as ``Leopards have wings.'' 
This difference is likely driven by greater uncertainty about the prevalence of the properties ``carries malaria'' and ``has manes'' as compared with ``has wings''.
``Carries malaria'' and ``has manes'' produced a more uncertain distribution of responses \red{(cite concentration parameters here..)} compared with ``has wings.''








We use these estimates as the prevalence of property for the target category in the model. 





 
 
 
 
 
 
 
 
 
 
 
 
 
 
 
 
 
 
 
 
 
 
 
 
 
 
In the experiments that follow, we explore the predictions of the $S_2$ model by comparing them to participant acceptability judgments of natural cases of generic utterances. 
We also verify the predictions of the $L_1$ model by comparing them to participants' interpretations of novel generic sentences.
These predictions are borne out by first eliciting the prior distribution on prevalence.
 

In a similar way, we posit a simple, scalar semantics for generics in which they express that the probability of the property given the category-----i.e. the property's \emph{prevalence}-----is above a threshold (cf. \citeA{Cohen1999}). We treat this threshold as unknown property of the language and thus, as a variable that must be reasoned about in context. 

It might seem paradoxical that vague language should get so much usage. 
Shouldn't speakers want to express their ideas as clearly as possible?
Language evolutionary pressures suggest that, to the contrary, such underspecification is in fact useful, given that there is some context through which the uncertainty can be resolved \cite{Piantadosi2012}.
In this work, context takes the shape of a listener and speaker's shared beliefs (i.e. common ground) about the property in question. This, coupled with 
standard inferences from conversational pragmatics, allows the listener to figure out the meaning of the otherwise vague utterance.
We extend the \emph{Rational Speech-Acts} (RSA) framework \cite{Frank2012,Goodman2013} to consider how a speaker determines what generics are acceptable to produce and how a listener might interpret generics differently depending on her belief's about the property in question. 
This formalism gives a new, computational perspective on how ideas are conveyed and how beliefs play a central role in understanding language.

\subsection{The phenomena}

One important test for a theory of generic meaning is that it captures the intuition shared by language users about the truth of certain generic sentences. To test the predictions of this model, we must first measure participants' prior expectations about the prevalence of the properties in question. Critically, however, we must measure not only prior expectations about the prevalence of the property for the category \emph{alone} but the distribution of expectations across categories (i.e. not only the probability of laying eggs for a robin, but also the probability of laying eggs for a cow and other animal species).  
We show how prevalence within the category alone is insufficient to explain generic acceptability, but a model that reasons pragmatically about the distribution over prevalence is sufficient to explain the variability in truth judgments.

Generic statements are not only puzzling because of their variable truth conditions, but also because of how they are interpreted. \citeA{Gelman2002} found that listeners  interpret novel facts about animals in the form of generics (e.g. ``Bears like to eat ants.'') as applying to nearly all of the category. Interpretations of bare plurals of novel animals with accidental properties (e.g. ``Morseths have wet fur.'') were found to be much weaker \cite{Cimpian2010}. These differences in interpretation are thought to be a result of different beliefs about the properties in question. This intuition is readily formalized in our language understanding model. Beliefs about properties are measured in independent experiments to serve as a foundation for the model's predictions. 
We compare our model's predictions to experiments examining the acceptability of naturalistic generics and interpretations of generics about novel categories.
\section{Experiment 1: Truth value judgments}

The lifted-threshold RSA model specified in Eq.~\ref{eq:S2} predicts probabilities of producing a generic given a prior distribution on the prevalence of the property. 
In Expt.~1a, we measure the distribution on prevalence of certain target properties by asking participants about the percentage of a kind with the property \footnote{So as to not bias the measurement in favor of a small number of stipulated animal categories, we ask participants to generate their own animal categories.}. 
In Expt.~1b, we measure the acceptability of a number of generic statements about the properties measured in Expt. 1a. 
We show how the prevalence within a property alone is insufficient to explain the diversity of truth judgments of these generics, but how a model of a pragmatic speaker who considers the distribution on prevalence can explain this wide range of truth judgments.



\section{General discussion}

The lifted-threshold RSA model presented in this paper takes a generic statement to be vague. ``K has F'' means ``many members of K have F, \emph{relative to other categories, the vast majority of which, very few or none of the individuals in those categories have F}''. 
The model predicts a range of truth judgments for generics that seem to not have to do with prevalence. 
By considering not only the prevalence within the category but prevalence across categories, the model is able to arrive at graded and property-sensitive predictions about the truth judgments of generic statements, as seen in Expt.~1. 
The model very naturally accounts for the parallel problem of generic interpretation. 
Different \emph{a priori} beliefs about the distributions of properties both within- and across-categories give rise the puzzling asymmetries in how generics are interpreted relative to what is needed to felicitously use the generic. 

We present a context-invariant semantics for generic statements. 
The stable meaning of a generic is as simple as possible: a threshold on prevalence, where the threshold is uncertain \emph{a priori} and only resolved by pragmatic reasoning.
This simple semantics is consistent with the profound phenomenon from cognitive development: Generics are learned at a very young age \cite{Gelman1998, Gelman2004, Gelman2008, Cimpian2008}.

We have explained two outstanding puzzles in language understanding. 
The first is that generic utterances can be true for a wide range of prevalence levels: from tigers with stripes all the way down to mosquitos with West Nile Virus. 
How can the prevalence of the property alone explain what makes some generics good things to say and others not as good? 
The insight is that with an uncertain semantics, the likely meaning of the generic (i.e. the likely threshold for acceptance) is inferred using listeners' \emph{a priori} beliefs and the communicative force of a speech act. 
This leads to different thresholds for different types of properties. 

The second phenomenon is the generic utterances often carry strong implications, though not all bare plurals have this impact. 
We measured participants' beliefs about the distributions of properties and found the accidental properties are utterly distinct in shape from biological properties.
Biological properties are characterized by \emph{widespread prevalence}, even though they may be rare across kinds (e.g. purple feathers).
Accidental properties, by contrast, are both rare across and within kinds: \emph{broken legs} are \emph{a priori} most likely to present in a small fraction of the population.
Thus, when given a vague utterance in the form a generic, the most likely \emph{a posteriori} prevalences vary in ways that match human participants' intuitions about the implied prevalence.


\subsection{Conceptual distinctions and prevalence}

We have shown that the truth conditions and implications of generics can be explained by beliefs about the prevalence of properties within- and across- categories. 
However, there is an open question of where \emph{these distributions} come from. 
It is quite plausible that these distributions are derived from higher-order conceptual knowledge about the nature of these properties \cite{Gelman2005, Keil1992}.

In Expt.~1a, participants estimated the prevalence of properties for many different kinds of animals. This was aggregated to form a distribution of prevalence across kinds.
In Expt.~2a, we measured this distribution by sequentially asking questions at different levels of abstraction: Question 1 was about the prevalence across categories while Question 2 was about prevalence within categories. 
It is likely that further abstracting the problem to more conceptual level questions (e.g. ``Are there differences between male and female lorches?''; ``Do you think a young fep is likely to have long legs?'') would elucidate the connection between distributions on prevalence and conceptual representations. 

Much of the psychological and philosophical work has looked beyond prevalence and focused on conceptual distinctions among generics \cite{Prasada2013, Leslie2008}. For example, \citeauthor{Prasada2013} has argued for a distinction between \emph{characteristic} properties (e.g. ``Diapers are absorbent.'') and \emph{statistical} properties (e.g. ``Diapers are white.''). Where in the prevalence-based semantics could such conceptual distinctions come into play?

Probabilistic models are a useful way to represent rich, structured knowledge of world \cite{Goodmanconcepts}. It's plausible that the prevalence distribution we've focused on in this work is actually derived from richer conceptual knowledge. For the purpose of the semantics of generics, this work shows that prevalence is sufficient to capture the range of truth judgments for these sentences. How a listener arrives at an estimate of the prevalence, or the prevalence distribution at large, may be the result of a probabilistic, conceptual model of the world. 

It's important to stress that our approach puts at its core listeners' \emph{beliefs} about the prevalence of the properties in question. 
Our participants' estimated the prevalence of certain properties to be much higher than the actual statistics of the world \red(example?).
This is an important methodological contrast to what is often employed in semantic approaches to generics, that prevalence is about \emph{actual prevalence in the world}. 
Along similar lines, this work used utterances about animal kinds in these experiments because participants' beliefs about these properties and categories are likely to be relatively homogenous (thus, lower noise). However, generic language is often used in everyday conversation to talk about social categories: In these domains, we would expect large individual differences in generic acceptability and interpretations.


An additional issue, one that likely interacts with conceptual distinctions, is how to determine the kinds of things against which a speaker implicitly compares the prevalence of the category in question. Here, we have used only generic sentences about animals, where the likely contrast class is other animals (e.g. ``Mosquitos carry malaria, \emph{relative to other kinds of living creatures}''). Moving outside the domain of animals, the makeup of this contrast class becomes less clear. 


A further observation concerning the nature of this contrast class is that focus, as a result of prosodic cues for example, can change the distribution against which one compares the prevalence of a property. For example, is a person asks ``What carries malaria?'', a very natural answer would be ``MOSQUITOS carry malaria'', since Mosquitos more than any other animal kind carry malaria. By contrast, if a person asks ``What are mosquitos like?'', is it natural to reply ``Mosquitos CARRY MALARIA''? 
Our suspicion is that it is not as natural because of salient alternatives (``Mosquitos bite you'', ``Mosquitos suck blood.'', ``Mosquitos make buzzing sounds''). 
Here, the contrast class is not with respect to \emph{other animals} but with respect to other properties \emph{of that animal}. 
Notice, however, that the underlying dimension that these properties would be compared along is still prevalence. What has changed is what constitutes this distribution: prevalence across categories vs. prevalence across properties.

\subsection{Resilience to counter-examples}

Another main staple of generic utterances is that they are resilient to counter-examples. 
For example, observing a few (perhaps, well-behaved) dogs that do not bark does not falsify ``Dogs bark.'' 
If generic statements are indeed vague, this phenomenon is very similar to the Sorites' Paradox, wherein a vague utterance and a plausible inductive premise can lead to unintuitive conclusions.

To see the similarity, consider the Sorites Paradox for ``The Empire State Building is tall.'' The paradox comes out when the following argument is considered:

\begin{quotation}

The Empire State Building is tall.

Any building 0.5 meters shorter than a tall building is still tall.

Therefore, a single-story building is tall.

\end{quotation} 

The first premise is true with a high probability. 
The second premise, is true with high probability but becomes less plausible the shorter in height one's reference gets. 

For generics, the version of the Sorites would look like

\begin{quotation}
Dogs bark. 

If we observe one dog that doesn't bark, then we could still say ``Dogs bark''. 

Therefore, if there were no dogs that barked, we would still conclude ``Dogs bark''.
\end{quotation} 

One thing to notice is that generics' resilience to counter-examples bears a strong similarity to the traditional inductive premise of the Sorites. 
It may be that the this feature of generics falls out of the inherent uncertainty of the meaning (i.e. its vagueness).

It's also interesting to consider the conclusion of this generic Sorites. 
This takes a very similar form of classic thought-experiments in the linguistics literature (e.g. ``Even t.



\subsection{What are generics?}

Intuitions about what qualifies as a generic statement have led to the consensus that there are generic and non-generic meanings that can be derived from similar syntactic forms. For example,
\begin{quotation}
	(1) Tigers are massive. 
	
	(2) Tigers are on the front lawn.
\end{quotation}

Classically, (1) is understood as referring to the kind whereas (2) is understood as referring to a plurality of instances. 

In our follow-up to \citeauthor{Cimpian2010}'s study on the implied prevalence of generic statements about accidental properties, we found a gradient of implied prevalence, which was driven by listener's \emph{a priori} beliefs about the prevalence of the property. 
The finding that different properties receive variable interpretations suggests a categorical distinction between generic and non-generic bare plurals may not be necessary. 
Consider the following 

\begin{quotation}
	(3) Tigers are on front lawns. 
	
	(4) Tigers are in zoos.
	
	(5) Tigers are in open grasslands.
\end{quotation}

Many would argue that (3) is not generic. At the same time, it does seem to implicate more tigers than (2). The same can be observed by comparing (4) to (3). Indeed, (5) has the intuitive appeal of applying to a large chunk of the category. (5) could be interpreted generically. 

It seems that as you go from (3) - (5), the likely implied prevalence also increases, to the point where (5) might apply to category as a whole. 
With a generic whose meaning is vague, as we propose here, listeners need not make categorical distinctions to interpret a sentence. 
Rather, they only need consult their beliefs about the properties in question to resolve reasonable interpretations. 

\subsection{Vagueness and communication}

We have presented a semantics for generics in which they convey a threshold on prevalence which is uncertain but fixed through pragmatic reasoning about the distribution of prevalence for the property in question.
In Expt.~2a, we showed how this prevalence distribution can be elicited by sequentially asking about the prevalence across- and within- categories. 
Within-category prevalence was elicited by asking a question about generalization: ``Imagine there is a glippet that has green legs. What \% of glippets do you think have green legs?''.

The model of generic communication that we've introduced uses a threshold whose value is uncertain. 
In a sense, the bare minimum that the generic communicates is that we are in a situation where there is a glippet that has green legs. 
The listener then is then left to her own devices to generalize as widely as she deems based on her beliefs about the property. 
This is consistent with the notion of a GEN operator that tells the listener to generalize this fact \cite{Leslie2008}.


The generic, it seems, doesn't convey any additional information beyond what the listener already knew about the prevalence of the property.
This shouldn't surprise us. ``John is tall'' does not actually tell us about what \emph{tall} means\footnote{However, ``John is a person'' does tell a listener about what \emph{tall} means in ``John is tall''.}. 
Rather, the listener is expected to come to the conversation with some beliefs about heights, and knowing that John is a person, be able to infer likely meanings for \emph{tall}.


\subsection{Conclusion} 


We have explored and demonstrated the viability of a scalar semantics for generics when coupled with a sophisticated pragmatics. 
A lower-bound threshold on prevalence---the probability of the property given the category---is inferred as part of pragmatic interpretation, yielding vague and context sensitive meanings. 

We formalized reasoning about the threshold in a lifted-threshold Rational Speech Acts model. This model predicted graded truth judgements and an asymmetry between truth and prevalence judgments. It also naturally accommodates the role of context, explaining these effects as the result of variation in the prevalence prior. 

Generics are ubiquitous in natural language. It might seem paradoxical, then, that the semantics of generic statements are underspecified. Why should vague language get so much usage? One possibility is apparent in the lifted-variable RSA model: generic language provides interlocutors with the flexibility to convey meanings as rich as our conceptual knowledge. %, which are easily understood in context. 
Generics are vague, but predictable and useful.

