\documentclass{article}
\usepackage{outlines} 
\setlength{\topmargin}{-.5in} 
\setlength{\textheight}{9in}
\setlength{\oddsidemargin}{.125in} 
\setlength{\textwidth}{6in}

 \begin{document}

\title{Generics can be vague}
\author{MHT, NDG\\CogSci 2015 Outline} \maketitle
 
\begin{outline}
  \0 The semantic puzzle of generics. Use examples to show how the truth conditions can have wide variability. 
  	\0 Cimpian, Brandone and Gelman (2010) found evidence for 2 phenomena 
		\1 Interpretation 1: Context can change the truth conditions for a generic
			\2 Replicate CBG Exp 1. 
			\2 Assume generic had a fixed-threshold, that varied by context alone.
				\3 What is that threshold? \emph{Bayesian Data Analysis}
				\3 Posterior predictive of fixed-threshold model
			\2 Posterior predictive should be pretty bad.
			\2 Guessing $\phi$ parameter pretty high
		\1 Interpretation2: The meaning is inferred from context (Lassiter and Goodman, 2014)
			\2 Bayesian-Language-Understanding Model: lifted-variable model of generics	
			\2 Some sort of bayesian analysis
	
	
 		\2 Evidence required for accepting a generic is different from that which the generic implies
			\3 \emph{Bayesian Language-Understanding Model: RSA with QUDs and roles in communication}
			\3 This asymmetry was not present for the quantifier ``most''
			\3 Degen and Goodman (2014 cogsci) found evidence that different dependent measures may map onto different communicative roles, formalized by speakers and listeners in RSA
			\3 Q: Can the asymmetry of the generic (and the symmetry of ``most'') be explained by speaker and listener in RSA?
			\3 Experiment: replicate CBG experiments 1 \& 2 (and be explicit about the analysis of ``some'', which CBG was not)
			\3 Model: RSA with QUDs and speaker/listener for the 2 tasks
				\4 \emph{Truth conditions} task $\rightarrow$ QUD = ``quantifier is true?'', task is Speaker (S1 or S2)
				\4 \emph{Implied prevalence} task $\rightarrow$ QUD = ``how many?'', task is Listener (L1)
			\3 Include prior elicitation experiment?

			
 \end{outline}
 \end{document}