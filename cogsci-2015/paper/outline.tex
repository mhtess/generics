\documentclass{article}
\usepackage{outlines} 
\setlength{\topmargin}{-.5in} 
\setlength{\textheight}{9in}
\setlength{\oddsidemargin}{.125in} 
\setlength{\textwidth}{6in}

 \begin{document}

\title{Generics}
\author{Michael Henry Tessler\\CogSci 2015 Outline} \maketitle
 
\begin{outline}
  \0 The semantic puzzle of generics
  	\1 Cimpian, Brandone and Gelman (2010) found evidence for 2 phenomena 
 		\2 Evidence required for accepting a generic is different from that which the generic implies
			\3 \emph{Bayesian Cognitive Model: RSA with QUDs and roles in communication}
			\3 This asymmetry was not present for the quantifier ``most''
			\3 Degen and Goodman (2014 cogsci) found evidence that different dependent measures may map onto different communicative roles, formalized by speakers and listeners in RSA
			\3 Q: Can the asymmetry of the generic (and the symmetry of ``most'') be explained by speaker and listener in RSA?
			\3 Experiment: replicate CBG experiments 1 \& 2 (and be explicit about the analysis of ``some'', which CBG was not)
			\3 Model: RSA with QUDs and speaker/listener for the 2 tasks
				\4 \emph{Truth conditions} task $\rightarrow$ QUD = ``quantifier is true?'', task is Speaker (S1 or S2)
				\4 \emph{Implied prevalence} task $\rightarrow$ QUD = ``how many?'', task is Listener (L1)
			\3 Include prior elicitation experiment?
	\2 Context can change the evidence required for accepting a generic (speaker task)
		\3 \emph{Bayesian Data Analysis}
		\3 Using t-tests seems crude for this; doesn't directly get at the question we're interested in
		\3 Since we've shown how the Bayesian model captures intuitions about semantics and interpretations, why don't we infer the threshold-values used in the model?
		\3 Experiment: replicate CBG experiment 4 for quantifier X context manipulation
		\3 Bayesian Data Analysis: infer threshold of quantifiers across different contexts
			\4 Hopefully find that generic has different threshold depending on the context, whereas ``most'' and ``some'' do not
	\1 What do generics and adjectives have in common?
		\2 the meaning is inferred from context (Lassiter and Goodman, 2014)
		\2 Model: lifted-variable model of generics
			
 \end{outline}
 \end{document}